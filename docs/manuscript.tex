% 
% Annual Cognitive Science Conference
% Sample LaTeX Paper -- Proceedings Format
% 

% Original : Ashwin Ram (ashwin@cc.gatech.edu)       04/01/1994
% Modified : Johanna Moore (jmoore@cs.pitt.edu)      03/17/1995
% Modified : David Noelle (noelle@ucsd.edu)          03/15/1996
% Modified : Pat Langley (langley@cs.stanford.edu)   01/26/1997
% Latex2e corrections by Ramin Charles Nakisa        01/28/1997 
% Modified : Tina Eliassi-Rad (eliassi@cs.wisc.edu)  01/31/1998
% Modified : Trisha Yannuzzi (trisha@ircs.upenn.edu) 12/28/1999 (in process)
% Modified : Mary Ellen Foster (M.E.Foster@ed.ac.uk) 12/11/2000
% Modified : Ken Forbus                              01/23/2004
% Modified : Eli M. Silk (esilk@pitt.edu)            05/24/2005
% Modified : Niels Taatgen (taatgen@cmu.edu)         10/24/2006
% Modified : David Noelle (dnoelle@ucmerced.edu)     11/19/2014
% Modified : Roger Levy (rplevy@mit.edu)     12/31/2018



%% Change "letterpaper" in the following line to "a4paper" if you must.

\documentclass[10pt,letterpaper]{article}

\usepackage{cogsci}

\cogscifinalcopy % Uncomment this line for the final submission 


\usepackage{pslatex}
\usepackage{apacite}
\usepackage{tikz}
\usepackage{float} % Roger Levy added this and changed figure/table
                   % placement to [H] for conformity to Word template,
                   % though floating tables and figures to top is
                   % still generally recommended!

%\usepackage[none]{hyphenat} % Sometimes it can be useful to turn off
%hyphenation for purposes such as spell checking of the resulting
%PDF.  Uncomment this block to turn off hyphenation.


%\setlength\titlebox{4.5cm}
% You can expand the titlebox if you need extra space
% to show all the authors. Please do not make the titlebox
% smaller than 4.5cm (the original size).
%%If you do, we reserve the right to require you to change it back in
%%the camera-ready version, which could interfere with the timely
%%appearance of your paper in the Proceedings.



\title{Errorless irrationality: removing prediction errors from the inverse base-rate effect}
 
\author{{\large \bf Lenard Dome (lenarddome@gmail.com)} \\
  Brain Research and Imaging Centre \\
  University of Plymouth, Research Way, Plymouth, PL6 8BU
  \AND {\large \bf Andy J. Wills (andy.wills@plymouth.ac.uk)} \\
  Brain Research and Imaging Centre \\
  University of Plymouth, Research Way, Plymouth, PL6 8BU}


\begin{document}

\maketitle

\begin{abstract}
Include no author information in the initial submission, to facilitate
blind review.  The abstract should be one paragraph, indented 1/8~inch on both sides,
in 9~point font with single spacing. The heading ``{\bf Abstract}''
should be 10~point, bold, centered, with one line of space below
it. This one-paragraph abstract section is required only for standard
six page proceedings papers. Following the abstract should be a blank
line, followed by the header ``{\bf Keywords:}'' and a list of
descriptive keywords separated by semicolons, all in 9~point font, as
shown below.

\textbf{Keywords:} 
irrationality; prediction error; inverse base-rate effect; categorization; contingency learning
\end{abstract}


\section{Introduction}

Inverse base-rate effect \cite<IBRE, >{medin1988problem} is an irrational tendency in humans to overweigh rare events when faced with ambiguity.
In a traditional design, people learn to categorise two overlapping sets of features under two distinct labels.
These sets share a single feature, A, and posess a unique feature, B and C, predictive of their respective category label.
During learning, these sets of features occur at different frequencies.
The features under the common label usually occur three times as much as features under the rare label \cite{kruschke1996base}.
Following training, people categorise features presented by themselves and un unique combinations.
People optimally label uniquely predictive features, B and C, presented individually with teir respective common and rare label.
Responses on the shared feature A also show base-rate following.
But when uniquely predictive features are paired, B and C, people tend to respond with the rare category label.
According Classical Probability Theory, the rational response is to attribute the common label to this ambiguous compound, because it is the most frequently occuring label.
This rare bias on ambiguous combinations of BC have been observed across different variety of different experiments and manipulations \cite{kalish2001inverse,don2017effects,don2017effects,inkster2022effect,wills2014attention}.
For a more thourough introduction into this irrational bias, see a review by \citeA{don2021attention}.

\subsection*{Assumptions of models of the IBRE}

The most prominent theories of the inverse base-rate effect involve an attentional mechanism that drives not only learning but responding as well.
These models are EXIT \cite{kruschke2001toward}, a three-layer neural network with competitive attentional gating and a four-layer neural network with an additional rapid attentional shift \cite{paskewitz2020dissecting}.
All these explanations rely on a process that reallocates attention in response to prediction errors.
Their explanation is simple.
During learning, people learn to label the AB compound first.
They are still learning to label the AC compound, so when they make an error, attention relocates towards the uniquely predictive feature C to reduce future errors.
This results in C acquiring higher attentional salience than B.
When the ambiguous BC compound is presented, C will dominate responding, resulting in an irrational tendency to respond with the rare label.
According to these models, this irrationality results from an optimisation process that tries to reduce the errors people make.

\subsection*{Current Study}

In this work, we intend to test this basic assumption of models of the IBRE.
In the following two experiments, we will gradually remove components from the design traditionally associated with prediction error.
In our first attempt, building on the observational learning condition of Experiment 2 in \citeA{johansen2007paradoxical}, we implement the canonical IBRE design with a caveat that category labels are presented in unison with features.

In our second attempt, we further remove the causal relationship between features and category labels.
The goal was to remove any design component that might affect attentional allocation.
Any presumption of causal relationship can inadvertantly relacote attention in line with the direction of causality between features and labels.

\subsection*{Related Work}

We will build on work carried out by \citeA{johansen2007paradoxical}.
\citeA{johansen2007paradoxical} failed to observe the inverse base-rate effect in an observational learning condition, where category labels were presented with features at the same time to participants.
They largely interpreted their results in opposition to EXIT \cite{kruschke2001toward}, but their experimental procedure does not correspond to the standard inverse base-rate effect procedure.
Their design involves disjoint-cues (where categories shared no features in common), while the canonical design depends on a shared features during training that facilitates attentional relocation, which in turn moves responding towards the irrational rare preference.
The only instance when they observed a rare bias was when common features presented in compound during training were paired with a rare feature presented by itself during training.
A simple explanation for that result is that the rare feature developed stronger connections with the category label.
Compound features share the connection with the category label, so any update to these connections dissipate between them.
There is no need to relocate attention to reduce errors, therefore attention will not push responding towards the rare label.

\section{Experiment 1}

Below, we detail our first attempt to test whether we could observe the rare response bias without any explicit error-driven psychological mechanism.


\subsection{Method}

\begin{table}[!ht]
  \begin{center}
    \caption{Abstract design of Experiment 1 including both test and training phases. \\}
    \label{tab:abstract-exp1}
    \begin{tabular}{llr} % text alignments
      \textbf{Training (Relative Frequencies)} & \textbf{Test}& \\
      \hline
      & \\
      $AB \to common_{1}$ (x 3) &  A, B, C,         &  \\
      $AC \to rare_{1}$   (x 1) &  AB, AC, BC      & x 20 \\
      \hline
    \end{tabular}
  \end{center}
\end{table}

\subsection{Results and Discussion}

\section{Experiment 2}

\subsection{Method}

\begin{table}[!ht]
  \begin{center}
    \caption{Abstract design of Experiment 2 including both test and training phases. X and Y are in place of the category labels. During the test phase, participants needed to select either X or Y to complete the features shown below.\\}
    \label{tab:abstract-exp2}
    \begin{tabular}{llr} % text alignments
      \textbf{Training (Relative Frequencies)} & \textbf{Test}& \\
      \hline
      & \\
      ABX x 3 &  A, B, C,         &  \\
      ACY x 1 &  AB, AC, BC      & x 20 \\
      \hline
    \end{tabular}
  \end{center}
\end{table}

\subsection{Results and Discussion}

\section{Discussion}

- auxiliary phenomenon (Wills CIRP addition)
- Things that any theory of IBRE should explain
- current theories fall short

\section{Open Science}

\section{Acknowledgments}

In the \textbf{initial submission}, please \textbf{do not include
  acknowledgements}, to preserve anonymity.  In the \textbf{final submission},
place acknowledgments (including funding information) in a section \textbf{at
the end of the paper}.

\bibliographystyle{apacite}

\setlength{\bibleftmargin}{.125in}
\setlength{\bibindent}{-\bibleftmargin}

\bibliography{library}


\end{document}
