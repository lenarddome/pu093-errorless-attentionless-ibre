% 
% Annual Cognitive Science Conference
% Sample LaTeX Paper -- Proceedings Format
% 

% Original : Ashwin Ram (ashwin@cc.gatech.edu)       04/01/1994
% Modified : Johanna Moore (jmoore@cs.pitt.edu)      03/17/1995
% Modified : David Noelle (noelle@ucsd.edu)          03/15/1996
% Modified : Pat Langley (langley@cs.stanford.edu)   01/26/1997
% Latex2e corrections by Ramin Charles Nakisa        01/28/1997 
% Modified : Tina Eliassi-Rad (eliassi@cs.wisc.edu)  01/31/1998
% Modified : Trisha Yannuzzi (trisha@ircs.upenn.edu) 12/28/1999 (in process)
% Modified : Mary Ellen Foster (M.E.Foster@ed.ac.uk) 12/11/2000
% Modified : Ken Forbus                              01/23/2004
% Modified : Eli M. Silk (esilk@pitt.edu)            05/24/2005
% Modified : Niels Taatgen (taatgen@cmu.edu)         10/24/2006
% Modified : David Noelle (dnoelle@ucmerced.edu)     11/19/2014
% Modified : Roger Levy (rplevy@mit.edu)     12/31/2018



%% Change "letterpaper" in the following line to "a4paper" if you must.

\documentclass[10pt,letterpaper]{article}

\usepackage{cogsci}

\cogscifinalcopy % Uncomment this line for the final submission 


\usepackage{pslatex}
\usepackage{apacite}
\usepackage{float} % Roger Levy added this and changed figure/table
                   % placement to [H] for conformity to Word template,
                   % though floating tables and figures to top is
                   % still generally recommended!

%\usepackage[none]{hyphenat} % Sometimes it can be useful to turn off
%hyphenation for purposes such as spell checking of the resulting
%PDF.  Uncomment this block to turn off hyphenation.


%\setlength\titlebox{4.5cm}
% You can expand the titlebox if you need extra space
% to show all the authors. Please do not make the titlebox
% smaller than 4.5cm (the original size).
%%If you do, we reserve the right to require you to change it back in
%%the camera-ready version, which could interfere with the timely
%%appearance of your paper in the Proceedings.



\title{Errorless irrationality: removing prediction errors from the inverse base-rate effect}
 
\author{{\large \bf Lenard Dome (lenarddome@gmail.com)} \\
  Brain Research and Imaging Centre \\
  University of Plymouth, Research Way, Plymouth, PL6 8BU
  \AND {\large \bf Andy J. Wills (andy.wills@plymouth.ac.uk)} \\
  Brain Research and Imaging Centre \\
  University of Plymouth, Research Way, Plymouth, PL6 8BU}


\begin{document}

\maketitle

\begin{abstract}
Include no author information in the initial submission, to facilitate
blind review.  The abstract should be one paragraph, indented 1/8~inch on both sides,
in 9~point font with single spacing. The heading ``{\bf Abstract}''
should be 10~point, bold, centered, with one line of space below
it. This one-paragraph abstract section is required only for standard
six page proceedings papers. Following the abstract should be a blank
line, followed by the header ``{\bf Keywords:}'' and a list of
descriptive keywords separated by semicolons, all in 9~point font, as
shown below.

\textbf{Keywords:} 
irrationality; prediction error; inverse base-rate effect; categorization; contingency learning
\end{abstract}


\section{Introduction}

- inverse base-rate effect
- investigating some strong assumptions about theories of the effect
- Johansen conceptual replication
- Removing error overall

\section{Experiment 1}

\subsection{Method}
\subsection{Results and Discussion}

\section{Experiment 2}

\subsection{Method}
\subsection{Results and Discussion}

\section{Discussion}

- auxiliary phenomenon (Wills CIRP addition)
- Things that any theory of IBRE should explain
- current theories fall short

\section{Open Science}

\section{Acknowledgments}

In the \textbf{initial submission}, please \textbf{do not include
  acknowledgements}, to preserve anonymity.  In the \textbf{final submission},
place acknowledgments (including funding information) in a section \textbf{at
the end of the paper}.

\bibliographystyle{apacite}

\setlength{\bibleftmargin}{.125in}
\setlength{\bibindent}{-\bibleftmargin}

\bibliography{CogSci_Template}


\end{document}
